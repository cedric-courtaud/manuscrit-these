\chapter{Résumé}

% Memory interferences may introduce important slowdowns in applications running on COTS multi-core processors.
% They are caused by concurrent accesses to shared hardware resources of the memory system.
% The induced delays are difficult to predict, making memory interferences a major obstacle to the adoption of COTS multi-core processors in real-time systems.
% In this article, we propose an experimental characterization of applications' memory consumption to determine their sensitivity to memory interferences.
% Thanks to a new set of microbenchmarks, we show the lack of precision of a purely quantitative characterization.
% To improve accuracy, we define new metrics quantifying qualitative aspects of memory consumption and implement a profiling tool using the \textsc{Valgrind} framework.
% In addition, our profiling tool produces high resolution profiles allowing us to clearly distinguish the various phases in applications' behavior.
% Using our microbenchmarks and our new characterization, we train a state-of-the-art regressor.
% The validation on applications from the \textsc{MiBench} and the \textsc{PARSEC} suites indicates significant gain in prediction accuracy compared to a purely quantitative characterization.

Les interférences du système mémoire peuvent entraîner d'importants ralentissements aux applications s'exécutant en parallèle sur les processeurs multi-cœurs COTS.
Elles ont pour origine les accès concurrents aux ressources matérielles partagées du système mémoire.
L'ampleur des retards causés par ce phénomène s'avère difficile à prédire, faisant des interférences un obstacle majeur à l'adoption des processeurs multi-cœur COTS dans les systèmes temps-réels.
Cette thèse est consacrée à la caractérisation de la sensibilité d'une application aux interférences mémoires à partir d'une caractérisation de son comportement exécutée seule.
Le but étant de pouvoir déterminer à priori si une application est sensible à ce problème ou non.
À l'aide d'un ensemble de microbenchmarks que nous avons préalablement introduit, nous montrons qu'une caractérisation purement quantitative du comportement d'accès à la mémoire caractérise la sensibilité aux interférences de façon très imprécise.
Afin de permettre une caractérisation plus précise de la sensibilité, nous introduisons différentes métriques permettant de quantifier des aspects quantitatifs de l'utilisation de la mémoire.
Afin de mesurer ces métriques, nous implémentons un prototype de profileur reposant sur des approches d'instrumentation binaire dynamique.
En plus de permettre la mesure des aspects qualitatifs, cet outil produit des profils haute résolution permettant de distinguer clairement les différentes phases dans les comportements applicatifs.
Enfin, nous utilisons les données issues de nos microbenchmarks pour entraîner un algorithme d'apprentissage automatique selon plusieurs caractérisations.
Les résultats expérimentaux montrent des réductions significatives de réduction d'erreur pour la prédiction du retard subi par des applications des suites \textsc{MiBench} et \textsc{PARSEC}.