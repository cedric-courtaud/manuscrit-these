\chapter{Abstract}

% Memory interferences may introduce important slowdowns in applications running on COTS multi-core processors.
% They are caused by concurrent accesses to shared hardware resources of the memory system.
% The induced delays are difficult to predict, making memory interferences a major obstacle to the adoption of COTS multi-core processors in real-time systems.
% In this article, we propose an experimental characterization of applications' memory consumption to determine their sensitivity to memory interferences.
% Thanks to a new set of microbenchmarks, we show the lack of precision of a purely quantitative characterization.
% To improve accuracy, we define new metrics quantifying qualitative aspects of memory consumption and implement a profiling tool using the \textsc{Valgrind} framework.
% In addition, our profiling tool produces high resolution profiles allowing us to clearly distinguish the various phases in applications' behavior.
% Using our microbenchmarks and our new characterization, we train a state-of-the-art regressor.
% The validation on applications from the \textsc{MiBench} and the \textsc{PARSEC} suites indicates significant gain in prediction accuracy compared to a purely quantitative characterization.

Interference from the memory system can cause significant slowdowns to applications running in parallel on COTS multi-core processors.
They are caused by concurrent accesses to shared hardware resources in the memory system.
The magnitude of the delays caused by this phenomenon is difficult to predict, making interference a major obstacle to the adoption of COTS multi-core processors in real-time systems.
This thesis is devoted to the characterization of the sensitivity of an application to memory interferences based on a characterization of its behavior in isolation.
The goal is to determine a priori if an application is sensitive to this problem or not.
Using a set of microbenchmarks that we have previously introduced, we show that a purely quantitative characterization of memory access behavior characterizes the sensitivity to interference in a very imprecise way.
In order to allow a more precise characterization of sensitivity, we introduce different metrics to quantify quantitative and qualitative aspects of memory use.
In order to measure these metrics, we implement a profiler prototype based on dynamic binary instrumentation approaches.
In addition to allowing the measurement of qualitative aspects, this tool produces high-resolution profiles that clearly distinguish the different phases in application behaviors.
Finally, we use data from our microbenchmarks to train a machine learning algorithm according to several characterizations.
Experimental results show significant reductions in error reduction for the prediction of the delay undergone by applications of the \textsc{MiBench} and \textsc{PARSEC} suites.
