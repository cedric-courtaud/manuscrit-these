
\chapter*{Publications}

% % Print your own papers.
% \begin{refsection}
% % If you print a bibliography within this section, only citations within this refsection will be printed.

% % Option 1: Make nocite for all of your papers.
% % Options 2 would be a seperate file which contains all of your papers.
% \nocite{courtaud2019improving}

% \nocite{courtaud2017compas}
% \nocite{courtaud2019compas}

% \defbibnote{myPrenote}{
% Les travaux présentés dans cette thèse on fait l'objet de la publication suivante :
% }
% \defbibnote{myPostnote}{
% }
% \printbibliography[
%     heading=bibintoc,
%     title={Author's Contributions},
%     prenote=myPrenote,
%     postnote=myPostnote
% ]
% \end{refsection}


Les travaux présentés dans cette thèse on fait l'objet des publications suivantes : 

\section*{Publications internationales}

\begin{itemize}
	\item Cédric Courtaud, Julien Sopena, Gilles Muller et Daniel Gracia Pérez. \textit{Improving Prediction Accuracy of Memory Interferences for Multicore Platforms.} 2019 40th IEEE International Real-Time Systems Symposium (RTSS'19).

	\item A. Blin, C. Courtaud, J. Sopena, J. Lawall, G. Muller. \textit{“Maximizing Parallelism without Exploding Deadlines in a Mixed Criticality Embedded System”}, 28th EUROMICRO Conference on Real-Time Systems (ECRTS'16), Toulouse, France (2016)

	\item A. Blin, C. Courtaud, J. Sopena, J. Lawall, G. Muller \textit{“Understanding the Memory Consumption of the MiBench Embedded Benchmark”}, Netys, Marakech, Morocco (2016)

\end{itemize}

\section*{Publications nationales}

\begin{itemize}
	\item Cédric Courtaud, Xavier Jean, Madeleine Faugère, Gilles Muller et Julien Sopena. \textit{Représentation spatiale et pseudo-temporelle des comportements mémoire d'une application.} Conférence d’informatique en Parallélisme, Architecture et Système (COMPAS'2017).

	\item Cédric Courtaud, Julien Sopena, Gilles Muller et Daniel Gracia Pérez. \textit{Caractériser l’impact des interférences mémoires dans les systèmes temps réel et multicoeur à partir des comportements applicatifs.} Conférence d’informatique en Parallélisme, Architecture et Système (COMPAS'2019). 
\end{itemize}